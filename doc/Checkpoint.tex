\documentclass[a4paper, 10pt]{article}
\usepackage{amsmath}
\usepackage{amssymb}
\usepackage{graphicx}
\usepackage{hyperref}
\usepackage{listings}
\usepackage{xcolor}
\usepackage{geometry}
\title{C Group Project -  Interim Report}
\author{Group 58}
\date{June 2024}

\begin{document}
\maketitle

\section{Introduction}
This project involves developing both an assembler to convert AArch64 source files into AArch64 binary files as well as an emulator to emulate the execution of these binary files on a Raspberry Pi. This report details the design and implementation of our emulator for Part I of the project.


\section{Group Coordination}
% Bullet Point 1
Initially, we all came together to develop the first version of the emulator, in which we made sure to incorporate all the basic functions. Of course, the emulator didn't work perfectly at first and passed very few tests, but with the framework in place we could easily split up the functions between us. This allowed each member to put the majority of their focus on their few parts while giving each of us a base understanding of how the emulator operated as a whole, giving us better comprehension of how our functions should operate to make the emulator operational.


\section{Group Dynamics}
% Bullet Point 2
So far, the group has been working well together. Each member has been contributing to the project and respecting our internal deadlines. However, as we move onto the assembler we predict an increase in challenge and so plan to meet more often to discuss how the code is developing and if any members need help with their roles.

\section{Architecture}

\subsection{Memory Management}
The emulator simulates a memory of 2MB, represented as an array of 32-bit words.

\begin{lstlisting}[language=C, caption=Memory Initialization]
#define MEMORY_SIZE (2 * 1024 * 1024) // 2MB of memory
uint32_t memory[MEMORY_SIZE / sizeof(uint32_t)];
\end{lstlisting}

\subsection{CPU State}
The CPU state is represented by a structure that includes the general-purpose registers, zero register, program counter, and processor state. The processor state includes the negative, zero, carry, and overflow flags (NZCV).

\begin{lstlisting}[language=C, caption=CPUState Structure]
typedef struct {
    uint64_t regs[31]; // General-purpose registers X0-X30
    uint64_t zr;       // Zero register
    uint64_t pc;       // Program Counter
    uint32_t pstate;   // Processor state (NZCV)
} CPUState;
\end{lstlisting}

\subsection{Initialization}
The CPU and memory are initialized to zero to reflect the state of an ARMv8 machine when it is first turned on.

\section{Emulator Reusability}
\begin{itemize}
    \item \textbf{Memory Initialization}: Using the same memory initialization format in both the emulator and assembler ensures their memory handling is aligned with each other.
    \item \textbf{Binary Loader}: The function to load binary data into memory can be adjusted to load and parse assembly source files into memory directly.
    \item \textbf{Instruction Decoder}: We can adapt and reuse the instruction processing logic from the emulator to translate assembly instructions into binary format for the assembler.
\end{itemize}


\section{Challenges}
We anticipate several challenges in building the assembler for the AArch64 source files. Namely, ensuring accurate translation of the assembly language into machine code and correctly handling labels and branches. To deal with these issues, we'll build our assembler starting with simpler instructions and progressively adding the more complex ones, testing the assembler at each stage to help pinpoint where our errors lie. We will also regularly review each other's code to help catch mistakes that we might have missed. This, along with reusing our proven code from the emulator should help minimise the amount of errors we make in the creation of the assembler.




\end{document}
