    \documentclass[a4paper, 10pt]{article}
\usepackage{amsmath}
\usepackage{amssymb}
\usepackage{graphicx}
\usepackage{hyperref}
\usepackage{listings}
\usepackage{xcolor}
\usepackage{geometry}
\usepackage{minted}
\usepackage[T1]{fontenc}
\usemintedstyle{xcode}
\newmintinline[clang]{c}{fontsize=\normalsize, breaklines}
\title{C Group Project - Final Report}
\author{Group 58}
\date{June 2024}

\begin{document}
\maketitle

%REMINDER TO DELETE, CHANGE AND FONT


\section{Introduction}
This project involves developing both an assembler to convert AArch64 source files into AArch64 binary files as well as an emulator to emulate the execution of these binary files on a Raspberry Pi. This report details the design and implementation of our assembler and the process for making the green LED blink on the Raspberry Pi.
\newline
\section{Assembler Implementation}
The assembler was designed to translate ARMv8 AArch64 assembly language into machine code. The assembler follows a two-pass approach to handle forward references to labels and ensure that all instructions are correctly encoded.

\subsection{Symbol Table}
The symbol table is implemented as an array of structures, where each structure stores a label and its corresponding address. We created functions to add labels to the table, check for the existence of a label, and retrieve the address associated with a label. This allows the assembler to efficiently manage and access label information during both passes.

\begin{minted}{c}
#define MAX_LABELS 1024
#define MAX_LINE_LENGTH 256

typedef struct {
    char label[MAX_LINE_LENGTH];
    int address;
} Label;

Label symbolTable[MAX_LABELS];
int labelCount = 0;
\end{minted}

\subsection{Two-Pass Assembly Process}

In the first pass, the assembler scans the entire assembly source file to build a symbol table. In the second pass, the assembler translates the assembly instructions into binary machine code by using helper functions designed for each instruction type to parse and encode the instructions. During this pass, it uses the symbol table created in the first pass to resolve the addresses of labels. This two-pass approach ensures that all label references can be correctly resolved, even if they appear before their definitions in the source file.

\subsection{Instruction Parsing}
The assembler breaks down each line of the source file into its constituent parts: the mnemonic (operation code) and its operands (registers, immediate values, or labels). The parsing function identifies these components and delegates the encoding process to specific functions designed to handle different types of instructions. Aliases are handled by these functions by redirecting to the same function after adjusting the operands.

\subsection{Binary File Output}

After both passes are completed, the assembler outputs the translated machine code to a binary file specified by the user. This binary file can be executed by an ARMv8 AArch64 emulator or directly on hardware that supports this architecture. The assembler handles file I/O operations to write the encoded instructions in the correct format, ensuring that the binary file is properly structured and ready for execution.
\newline
% WORRY ABOUT THIS LATER
\section{LED Blinking Implementation}
\subsection{Overview}
To make the LED blink on the Raspberry Pi, an A64 assembly program was written and assembled using our assembler. The program configures the GPIO pins and toggles the LED state in a loop. The assembled binary was executed on the Raspberry Pi to verify the functionality.

\subsection{Assembly Program}
The following assembly code compromises the main loop making the LED blink on the Raspberry Pi by toggling the LED state and managing delays:

\begin{minted}{c}
eor x10, x10, x10
loop:
  str w10, [x3]
  str w1, [x2]

eor x5, x5, x5
delay:
  add x5, x5, #1
  cmp x5, x6
  b.ne delay

str w10, [x2]
str w1, [x3]

eor x5, x5, x5
delay2:
  add x5, x5, #1
  cmp x5, x6
  b.ne delay2

b loop
\end{minted}

\subsection{Code Breakdown}
\begin{itemize}
    \item \textbf{Initialization}: Configures the GPIO control register and sets the GPIO pin as an output.
    \item \textbf{Setup GPIO Registers}: Loads the base addresses and offsets for the GPIO set and clear registers.
    \item \textbf{Main Loop}: Clears the GPIO pin to turn off the LED. Sets the GPIO pin to turn on the LED. Includes delay loops to create a visible blinking effect.
\end{itemize}

This assembly code continuously toggles the GPIO pin connected to the green LED on and off, creating a blinking effect.
\newline
\section{Challenges and Solutions}


\subsection{Challenges}
One major challenge was ensuring that each new instruction was correctly encoded into machine code. This required us to thoroughly understand how the instruction set is implemented. We also found it quite hard to find and debug the problems arising from edge cases in our codes.

\subsection{Solutions}
We tackled these challenges by incrementally adding and testing each new instruction. Regular code reviews and pair programming sessions helped identify and resolve issues early in the development process. Additionally, we created extensive test cases to ensure the new instructions worked correctly and to help us detect incorrect behaviour resulting from edge cases
% REWROTE
\newline
\section{Extension}
For the extension, we developed a relocating loader that allows code to be loaded at an offset specified in the header file. This required us to modify the emulator and assembler to handle the offset and addresses accordingly. This feature provides us with flexibility in how the programs are loaded into memory, giving us control over the exact location in memory that values are to be stored.

As this is an extension of the current code base, this extension is within the files in the src directory.
\subsection{Extension Testing}

We tested the relocating loader using the standard test suite. The tests provided us with the memory locations for expected and actual results, which allowed us to manually verify tests to ensure the relocating loader was functioning correctly. By comparing the starting from the offset in the actual results with the non-offset expected results, we confirmed that the loader adjusted the values correctly, ensuring that the code executes as intended from any memory location.

\section{Testing and Effectiveness}
The assembler was tested using the provided test suite and additional custom test cases using our emulator to ensure correct behaviour. Each type of instruction was tested for correct encoding and execution. The testing confirmed the assembler's accuracy and reliability in translating assembly source files into executable binaries. While the provided test suite covered a wide range of scenarios, it could not encapsulate all possible edge cases and variations we therefore implemented additional test cases to further validate the assembler's robustness.

\subsection{Testing Methodology}
We used unit tests for individual functions and integration tests for complete assembly programs. The test suite included a variety of programs dealing with different instruction types, edge cases and invalid inputs to ensure robust error handling which allowed us to pinpoint the issues with our assembler throughout the project.

\subsection{Effectiveness}
The testing showed that the assembler correctly handles all supported instructions and labels. The encoded binary files from our own test cases were verified by executing them in our emulator which confirmed that they produced the expected results.
\newline
\section{Group Reflection}
Working in a group provided us with valuable insights in developing software with peers. Communication and task division were crucial for the project's success and in particular, we found pair programming to be very effective in solving complex problems and spotting errors quickly. 

\subsection{Communication and Task Division}
To make sure each member had an equal amount of responsibility we held regular meetings to discuss progress, allocate tasks and reassign tasks that were deemed easier or harder than first assumed. Each member focused on different parts of the project, such as instruction parsing, symbol table management, and binary file output. This approach ensured balanced workload distribution and efficient problem-solving. 

\subsection{Future Improvements}
For future projects, we would increase the frequency of code reviews. This would help us catch errors much earlier and improve the overall quality of our code. Additionally, we plan to document our code more thoroughly to make it easier for others to understand and contribute to it.
\newline






%REWRITE AI
\section{Individual Reflections}
\subsection{Member 1: Omar Sawalha}
Working on this project has been a highly valuable experience for me. I had the opportunity to apply my knowledge extensively, particularly in assembly language and C programming. Reflecting on the Peer Assessment feedback and my other experiences, I felt like I fit in well into the group. I felt like I was quite good at communicating effectively with the rest of the group. Initially I was not used to working on the same coding project as others, and was not used to documenting my work in an effective and clear manner, however during this project I learnt to document my work clearly and concisely, which is something I will carry with me as I progress and work with different groups of people. 

\subsection{Member 2: Mokhtar Rekada}
This project has been extremely useful in advancing my skills, especially in low-level programming and collaborative teamwork. I learnt how to split large workloads effectively across a group, allowing each member to work towards goal by efficiently working on subsections of the programs. I fit in well with the group, and we all managed to work together very efficiently, especially while pair-programming. The ability to pair-program well is something I will use in the future while working on other projects.

\subsection{Member 3: Mohamed Omar}
Throughout this project, I found  in tackling the challenges of emulator development, which deepened my understanding of system architecture and honed my problem-solving skills. Reflecting on the team, I realize the importance of engaging in group discussions. In future collaborations, I aim to actively take part more in team conversations to effectively share insights and contribute more meaningfully to our solutions. 

\subsection{Member 4: Orvi Chowdhury}
My journey with this project has been immensely insightful, particularly in developing the LED blinking functionality and understanding the intricate software-hardware interaction. One crucial takeaway for me has been the importance of effective teamwork and communication within our group. I've realized the significance of clear documentation in ensuring everyone stays aligned and can contribute effectively. Moving forward, I am committed to enhancing my documentation practices to better support our team's collaborative efforts. Clear and concise comments will not only aid in understanding but also facilitate smoother future developments, ensuring our collective success in upcoming projects.


\end{document}
